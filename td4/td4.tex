% Written on Wed 28 Mar 2018 09:48:54 CEST
% by Jean-Baptiste Caillau, LJAD, Univ. Cote d'Azur & CNRS/Inria
\documentclass[11pt,a4paper]{article}
\usepackage{hyperref}
\usepackage{amsmath}
\usepackage{mathrsfs}
\usepackage{graphicx}
\usepackage[latin1]{inputenc}
\usepackage{tp}
\def\N{\mathbf{N}}
\def\Z{\mathbf{Z}}
\def\Q{\mathbf{Q}}
\def\R{\mathbf{R}}
\def\C{\mathbf{C}}
\def\K{\mathbf{K}}
\def\L{\mathrm{L}}
\def\H{\mathrm{H}}
\def\M{\mathrm{M}}
\def\W{\mathrm{W}}
\def\O{\mathrm{O}}
\def\tr{\mathrm{tr}}
\def\Vect{\mathrm{Vect}}
\def\iy{\infty}
\def\d{\mathrm{d}}
\def\t{\ \!^t\!}
\def\veps{\varepsilon}
\def\vphi{\varphi}
\def\la{\langle}
\def\ra{\rangle}
\def\noi{\noindent}
\def\cf{\emph{cf.}}
\def\ie{\emph{i.e.}}
\renewcommand{\tilde}{\widetilde}
\renewcommand{\hat}{\widehat}

\title{TD~4 -- Projection, orthogonalit\'e}
\shorttitle{TD~4}
\numero{TD~4}
\date{2019--2020}
\discipline{MI2}
\promotion{Polytech Nice Sophia --- MAM3}

\begin{document}
\maketitle

% Exercice 1
\begin{Exercice}
Soit $A = (a_{ij})_{1 \leq i,j \leq n} \in \M(n,\R)$, on
rappelle que $\tr(A)=\sum_{i=1}^n a_{ii}$.
\begin{Question}
Montrer que l'application $(.|.)$ de $\M(n,\R) \times
\M(n,\R)$ dans $\R$ d\'efinie par
$$ (X|Y) = \tr(\t XY) $$
d\'efinit un produit scalaire qui fait de $\M(n,\R)$ un espace euclidien.
La norme matricielle associ\'ee s'appelle la \emph{norme de Frobenius}.
\end{Question}
\begin{Question} Soient $X$, $Y$ et $A$ dans $\M(n,\R)$, montrer que ce
produit scalaire poss\`ede les propri\'et\'es suivantes~:
\begin{itemize}
  \item[(i)]  $(\t X|\t\,Y) = (X|Y)$
  \item[(ii)] $(AX|Y) = (X|\t AY).$
\end{itemize}
\end{Question}
\begin{Question} Soit $O \in \O(n,\R)$ une matrice orthogonale, montrer
que l'application $X \mapsto OX$ est une isom\'etrie de $\M(n,\R)$.
\end{Question}
\end{Exercice} \vspace*{1em}

% Exercice 2
\begin{Exercice}
On rappelle que $\L^2([0,2\pi],\R)$ muni du produit scalaire
$$ (x|y) = \int_{[0,2\pi]} xy\,\mathrm{d}t $$
est un espace de Hilbert.
Soit $F = \Vect(\{\sin,\cos\})$, et soit $x_0$ d\'efini par
$x_0(t)=e^t$. Justifier que $x_0$ poss\`ede un unique projet\'e
orthogonal sur $F$ et le calculer.
\end{Exercice} \vspace*{1em}

% Exercice 3
\begin{Exercice}
\begin{Question} Montrer que l'espace
$$ \ell^2=\{X=(x_n)_n \in \R^\N\ |\ \sum_n |x_n|^2<\iy\} $$
muni du produit scalaire $(X|Y) = \sum_n x_n y_n$
est un espace de Hilbert. \end{Question}
\begin{Question} Soit $G_k = \{X \in \ell^2\ |\ \sum_{n=0}^{k-1} x_n = 0\}$
(o\`u $k \geq 1$ est fix\'e). Justifier que $G_k$ est un sev ferm\'e de
$\ell^2$. \end{Question}
\begin{Question} Soit $X=(1,0,\dots,0,\dots) \in \ell^2$, \'evaluer
$d(X,G_k)$, la distance de $X$ \`a $G_k$. \end{Question}
\end{Exercice} \vspace*{1em}

\enlargethispage{1cm}
% Exercice 4
\begin{Exercice} R\'esoudre le probl\`eme d'optimisation
$$ \left\{ \begin{array}{l}
  \min \int_{[-1,1]} |x^5-a_4 x^4-a_3 x^3-a_2 x^2-a_1 x-a_0|^2\,\mathrm{d}x\\
  a=(a_0,\dots,a_4) \in \R^5.
\end{array} \right. $$
\end{Exercice} \vspace*{1em}

\vfill \begin{flushright}{\footnotesize \emph{En ligne sous}
\texttt{caillau.perso.math.cnrs.fr/mi2}} \end{flushright}

\end{document}
