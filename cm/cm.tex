% Written on Tue  6 Feb 2018 22:20:27 CET
% by Jean-Baptiste Caillau, Univ. Cote d'Azur & CNRS/Inria
\documentclass[11pt,a4paper]{article}
\usepackage{hyperref}
\usepackage{amsmath}
\usepackage{mathrsfs}
\usepackage[french]{babel}
\usepackage{graphicx}
\usepackage{tp}
\def\N{\mathbf{N}}
\def\Z{\mathbf{Z}}
\def\Q{\mathbf{Q}}
\def\R{\mathbf{R}}
\def\C{\mathbf{C}}
\def\DD{\mathscr{D}}
\def\CC{\mathscr{C}}
\def\T{\mathbf{T}}
\def\K{\mathbf{K}}
\def\L{\mathrm{L}}
\def\H{\mathrm{H}}
\def\W{\mathrm{W}}
\def\M{\mathrm{M}}
\def\O{\mathrm{O}}
\def\Im{\mathrm{Im}}
\def\Vect{\mathrm{Vect}}
\def\Min{\mathrm{Min}}
\def\BV{\mathrm{BV}}
\def\Isom{\mathrm{Isom}}
\def\iy{\infty}
\def\d{\mathrm{d}}
\def\t{\ \!^t\!}
\def\tr{\mathrm{tr}}
\def\veps{\varepsilon}
\def\vphi{\varphi}
\def\la{\langle}
\def\ra{\rangle}
\def\noi{\noindent}
\def\cf{\emph{cf.}}
\def\ie{\emph{i.e.}}
\def\etc{\emph{etc.}}
\renewcommand{\tilde}{\widetilde}
\renewcommand{\hat}{\widehat}
\theoremstyle{plain}
\newtheorem{thrm}{Th\'eor\`eme}[section]
\newtheorem{prpstn}{Proposition}[section]
\newtheorem{lmm}{Lemme}[section]
\newtheorem{crllr}{Corollaire}[section]
\newtheorem{dfntn}{D\'efinition}[section]
\theoremstyle{definition}
\newtheorem{rmrk}{Remarque}[section]

\title{Maths de l'ing\'enieur~2}
\shorttitle{Plan du cours}
\numero{Plan du cours}
\date{2019--2020}
\discipline{MI2}
\promotion{Polytech Nice-Sophia --- MAM3}

\begin{document}
\maketitle

\section*{I. Calcul diff\'erentiel}
\paragraph{1. D\'eriv\'ee}
\begin{itemize}
\item d\'erivabilit\'e locale et propri\'et\'es (unicit\'e, continuit\'e),
d\'erivabilit\'e globale
\item exemples fondamentaux~: applications lin\'eaires, affines, bilin\'eaires,
quadratiques
\item d\'erivation des fonctions compos\'ees
\item applications composantes
\end{itemize}

\paragraph{2. D\'erivation partielle}
\begin{itemize}
\item applications et d\'eriv\'ees partielles en un point, matrice jacobienne
\item \'equivalence en classe $\CC^1$ et contrexemple
\item th\'eor\`eme de Schwarz
\item gradient, hessien, laplacien
\item exemple fondamental~: forme quadratique
\end{itemize}

\paragraph{3. Accroissements finis} 
\begin{itemize}
\item th\'eor\`eme des accroissements finis (premi\`ere et deuxi\`eme forme)
\item th\'eor\`eme des fonctions implicites
\item algorithme de Newton
\end{itemize}

\section*{II. Espaces vectoriels norm\'es}
\begin{itemize}
\item norme, boules ouvertes et ferm\'ees
\item parties ouvertes, ferm\'ees, adh\'erence
\item suites, caract\'erisation s\'equentielle des parties ferm\'ees
\item continuit\'e, caract\'erisation s\'equentielle, cas des applications lin\'eaires
\item suites de Cauchy, parties compl\`etes, th\'eor\`eme du point fixe
\item parties compactes, cas de la dimension finie, image continue d'un compact
\end{itemize}

\section*{III. Espaces de Hilbert}
\paragraph{1. Produit scalaire}
\begin{itemize}
  \item produit scalaire, th\'eor\`eme de Cauchy-Schwarz, norme associ\'ee
  (Min\-kow\-ski)
  \item espace pr\'ehilbertien, hilbertien, euclidien
  \item exemples~: $\R^n$, $\M(m,n,\R)$, $\L^2(X,\mathscr{B},\mu)$, $\ell^2(\N)$
\end{itemize}

\paragraph{2. Th\'eor\`eme de la projection}
\begin{itemize}
  \item distance d'un point \`a une partie, projet\'e, partie convexe
  \item identit\'e du parall\'elogramme, th\'eor\`eme de la projection
  \item caract\`ere $1$-lipschitzien de la projection
\end{itemize}

\paragraph{3. Orthogonalit\'e}
\begin{itemize}
  \item orthogonal d'une partie, propri\'et\'es de base
  \item suppl\'ementaire orthogonal d'un sev ferm\'e
  \item double orthogonal
  \item base hilbertienne, Parseval
\end{itemize}

\paragraph{4. S\'eries de Fourier trigonom\'etriques}
\begin{itemize}
  \item base hilbertienne trigonom\'etrique sur $\L^2_{2\pi}(\R)$ (et $\L^2_{2\pi}(\C)$)
  \item r\'esultats de convergence compl\'ementaires, cas $\H^1_{2\pi}$
  et $\mathscr{C}^1$ par morceaux (Dirichlet)
  \item transform\'ee de Fourier discr\`ete, FFT
  \item extension au cas multidimensionnel  
\end{itemize}

\section*{IV. Formulation faible de pb aux limites}
\paragraph{1. Distributions}
\begin{itemize}
\item motivation~: solutions g\'en\'eralis\'ees de $xy'(x)=0$ 
\item ensemble $\DD(\Omega)$ (dimension un), exemple canonique de fonction plate
\item ensemble $\DD'(\Omega)$, notion d'ordre d'une distribution 
\item exemples~: (i) fonctions localement int\'egrable, (ii) valeur principale,
(iii) mesures
\item d\'erivation d'une distribution (motiver par le cas $\CC^1$), exemples
(Heaviside, Dirac)
\item r\'esolution de $T'=0$
\item produit par une fonction $\CC^\iy$, r\`egle de Leibniz associ\'ee
\item r\'esolution de $xT=0$
\item convergence (faible) dans $\DD'(\Omega)$, continuit\'e de la d\'erivation
\end{itemize}
\paragraph{2. Formulation faible en dimension un}
\begin{itemize}
\item motivation~: notion de solution faible pour $-u''+u=f$ sur $]0,1[$, conditions
de Neumann
\item d\'efinition de $\H^1(]0,1[)$ et propri\'et\'es (caract\`ere hilbertien,
injection $\CC^0([0,1])$, int\'egration par parties)
\item th\'eor\`eme de Riesz (compl\'ement du III)
\item d\'emarche variationnelle~:
\begin{itemize}
\item[A.]~Toute solution forte est solution faible
\item[B.]~Existence et unicit\'e de solution faible
\item[C.]~R\'egularit\'e de la solution faible
\item[D.]~Retour \`a une solution forte
\end{itemize}
\end{itemize}

\paragraph{Organisation et intervenants}

\begin{itemize}
  \item 12 s\'eances (1 s\'eance = 1H CM + 2H TD)
  \item J.-B.~Caillau (\verb+jean-baptiste.caillau@univ-cotedazur.fr+)
  \item L.~Monasse (\verb+laurent.monasse@inria.fr+)
\end{itemize}

\paragraph{\'Evaluation}
\begin{itemize}
  \item 2 EX CC (coeff.\ 1 tous les deux) 
  \item 1 EX terminal (coeff.\ 2)
\end{itemize}

%\paragraph{Questions de cours}

\paragraph{Bibliographie}
\begin{enumerate}
\item Br\'ezis, H.
{\em Analyse fonctionnelle, th\'eorie \& applications.} Dunod, 2005.
\item Exo7.
{\em Cours de math\'ematiques de premi\`ere ann\'ee.} \verb+exo7.emath.fr+
\item Gasquet, C.;  Witomski, P.
{\em Analyse de Fourier et applications.} Dunod, 2000.
\item Liret, F.; Martinais, D.
{\em Analyse~: cours de premi\`ere ann\'ee.} Dunod, 2003.
\item Monasse, D.
{\em Cours de Math\'ematiques pour les classes MP et MP$^*$.} Vuibert, 1997.
\item Schwartz, L.
{\em M\'ethodes math\'ematiques pour les sciences physiques.} Hermann, 1966.
\item Zuily, C.
{\em \'El\'ements de distributions et d'\'equations aux d\'eriv\'ees partielles.}
Dunod, 2002.
\end{enumerate}

\vfill \begin{flushright}{\footnotesize \emph{En ligne sous}
\texttt{caillau.perso.math.cnrs.fr/mi2}} \end{flushright}
\end{document}
