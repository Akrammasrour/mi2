% Written on Tue  4 Jun 2019 15:15:35 CEST
% by Jean-Baptiste Caillau, Universite Cote d'Azur, CNRS, Inria, LJAD
% todo: 
\documentclass[11pt,a4paper]{article}
\usepackage{hyperref}
\usepackage{amsmath}
\usepackage{mathrsfs}
\usepackage[french]{babel}
\usepackage{wasysym}
\usepackage{graphicx}
\usepackage{tp}
\usepackage{version}
\def\N{\mathbf{N}}
\def\Z{\mathbf{Z}}
\def\Q{\mathbf{Q}}
\def\R{\mathbf{R}}
\def\C{\mathbf{C}}
\def\CC{\mathscr{C}}
\def\DD{\mathscr{D}}
\def\T{\mathbf{T}}
\def\K{\mathbf{K}}
\def\L{\mathrm{L}}
\def\H{\mathrm{H}}
\def\W{\mathrm{W}}
\def\M{\mathrm{M}}
\def\O{\mathrm{O}}
\def\Im{\mathrm{Im}}
\def\Vect{\mathrm{Vect}}
\def\Min{\mathrm{Min}}
\def\BV{\mathrm{BV}}
\def\Isom{\mathrm{Isom}}
\def\atan{\mathrm{atan}}
\def\iy{\infty}
\def\d{\mathrm{d}}
\def\t{\ \!^t\!}
\def\tr{\mathrm{tr}}
\def\veps{\varepsilon}
\def\vphi{\varphi}
\def\vp{\mathrm{vp}}
\def\la{\langle}
\def\ra{\rangle}
\def\noi{\noindent}
\def\cf{\emph{cf.}}
\def\ie{\emph{i.e.}}
\def\etc{\emph{etc.}}
\renewcommand{\tilde}{\widetilde}
\renewcommand{\hat}{\widehat}
\theoremstyle{plain}
\newtheorem{thrm}{Th\'eor\`eme}[section]
\newtheorem{prpstn}{Proposition}[section]
\newtheorem{lmm}{Lemme}[section]
\newtheorem{crllr}{Corollaire}[section]
\newtheorem{dfntn}{D\'efinition}[section]
\theoremstyle{definition}
\newtheorem{rmrk}{Remarque}[section]

%\excludeversion{corr}
\includeversion{corr}

\title{Examen}
\shorttitle{Examen}
\numero{Examen}
\date{2018--2019}
\discipline{MI2}
\promotion{Polytech Nice-Sophia --- MAM3}

\begin{document}
\maketitle

{\bf Dur\'ee 2H00. Tous les exercices sont ind\'ependants.
Le bar\`eme pr\'e\-vi\-sion\-nel est indiqu\'e pour chaque exercice.
%Rendre sur des copies s\'epar\'ees l'exercice 1 d'une part, les exercices 2 et 3
%d'autre part.
Documents autoris\'es~: une feuille de notes de cours recto-verso manuscrite.}

% Exercice 1
\begin{Exercice}[4 points]
Montrer que l'application $f : \R^3 \to \R^2$ d\'efinie par
\[ f(x_1,x_2,x_3) := \left[ \begin{array}{c}
  (x_1+x_2)\exp(x_2-x_3)\\
  x_3\sin(x_1^2+x_2^2)
  \end{array} \right] \]
est d\'erivable et donner l'expression de sa d\'eriv\'ee. (Le symbole $\exp$
d\'esigne l'exponentielle.)

\begin{corr} $\RHD$ Les d\'eriv\'ees partielles de chaque composante de l'application
existent et sont continues, la fonction est donc de classe $\CC^1$ (et en particulier
d\'erivable). On a
\[ f'(x) = \left[ \begin{array}{ccc}
  \exp(x_2-x_3) & (1+x_1+x_2)\exp(x_2-x_3) & -(x_1+x_2)\exp(x_2-x_3)\\
  2x_1x_3\cos(x_1^2+x_2^2) & 2x_2x_3\cos(x_1^2+x_2^2) & \sin(x_1^2+x_2^2)
  \end{array} \right]. \]
\end{corr}
\end{Exercice} \vspace*{1em}

% Exercice 2
\begin{Exercice}[5 points]
Soit $f:\R \to \R$, d\'efinie par $f(t)=\pi/2-|t|$ sur $[-\pi,\pi[$, et prolong\'ee par
$2\pi$-p\'eriodicit\'e \`a tout $\R$.

\begin{Question} Donner l'expression de la s\'erie de Fourier de $f$
sur la base hilbertienne des polyn\^omes trigonom\'etriques.
\end{Question}

\begin{corr} $\RHD$ La fonction est paire donc tous les $b_n$ sont
nuls et
\[ a_{2p} = 0,\quad a_{2p+1} = \frac{4}{(2p+1)^2\sqrt{\pi}}\,,\quad p \geq 0. \]
\end{corr}

\begin{Question} En appliquant Dirichlet en un instant $t$ bien choisi,
d\'eterminer la somme de la s\'erie
\[ \sum_{p \geq 0} \frac{1}{(2p+1)^2} \cdot \] 
\end{Question}

\begin{corr} $\RHD$ En appliquant Dirichlet (la fonction est $\CC^1$ par morceaux)
en $t=0$ ou en $t=\pm\pi$ (points en lesquels la fonction est continue),
\[ \sum_{p \geq 0} \frac{1}{(2p+1)^2} = \frac{\pi^2}{8} \cdot \] 
\end{corr}

\begin{Question} En appliquant Parseval, d\'eterminer la somme de la s\'erie
\[ \sum_{p \geq 0} \frac{1}{(2p+1)^4} \cdot \]
\end{Question}

\begin{corr} $\RHD$ On trouve $\|f\|^2=\pi^3/6$, d'o\`u
\[ \sum_{p \geq 0} \frac{1}{(2p+1)^4} = \frac{\pi^4}{96} \cdot \]
\end{corr}

\end{Exercice} \vspace*{1em}

% Exercice 3
\begin{Exercice}[5 points]
On cherche les distributions $S$ dans $\DD'(\R)$ qui v\'erifient
\begin{equation} \label{eq1} 
  xS'=T_{x}
\end{equation}
o\`u $T_x$ d\'esigne la distribution r\'eguli\`ere telle que
$\la T_x,\vphi \ra = \int_\R x\vphi(x)\,\d x$.

\begin{Question} Calculer $xT_1$ (o\`u $T_1$ d\'esigne la distribution r\'eguli\`ere
associ\'ee \`a la fonction constante \'egale \`a $1$).
\end{Question}

\begin{corr} $\RHD$
\[ xT_1=T_x \]
\end{corr}

\begin{Question} En d\'eduire que  $S \in \DD'(\R)$ est solution de (\ref{eq1}) si et
seulement si
\[ x(S-T_x)'=0. \]
\end{Question}

\begin{corr} $\RHD$ On a $S$ solution si et seulement si $xS'=xT_1$,
c'est-\`a-dire si et seulement si (utiliser $(T_x)'=T_1$)
\[ x(S-T_x)'=0. \]
\end{corr}

\begin{Question} En d\'eduire que $S \in \DD'(\R)$ est solution de (\ref{eq1}) si et
seulement si
\[ (S-T_x-cT_H)' = 0 \]
o\`u $H=1_{\mathbf{R}_+}$ d\'esigne la fonction de Heaviside, et o\`u $c \in \R$ est
une constante.
\end{Question}

\begin{corr} $\RHD$ On sait que $x(S-T_x)'=0$ si et seulement s'il existe
$c \in \R$ tel que $(S-T_x)'=c\delta$~; comme $\delta=(T_H)'$, cette \'equation
se r\'e\'ecrit
\[ (S-T_x-cT_H)'=0. \]
\end{corr}

\begin{Question} En d\'eduire l'ensemble des solutions de (\ref{eq1}) dans $\DD'(\R)$.
\end{Question}

\begin{corr} $\RHD$ Les solutions sont donc les distributions
$S=T_x+cT_H+d$, avec $c$ et $d$ deux constantes arbitraires dans $\R$.
\end{corr}

\end{Exercice} \vspace*{1em}

% Exercice 4
\begin{Exercice}[6 points] On consid\`ere le probl\`eme avec conditions aux limites
\emph{p\'eriodiques} suivant~: trouver $u \in \CC^2([0,1])$ telle que
\[ -u''(t)+u(t) = f(t),\quad t \in ]0,1[, \]
\[ u(0)=u(1),\quad u'(0)=u'(1), \]
o\`u $f$ est une fonction donn\'ee de $\CC^0([0,1])$.

\begin{Question} On d\'efinit
\[ H := \{ u \in \H^1(]0,1[)\ |\ u(0)=u(1) \}. \]
Montrer que $H$ est un sous-espace vectoriel 
ferm\'e de $\H^1(]0,1[)$ comme noyau d'une forme
lin\'eaire continue que l'on pr\'ecisera.
\end{Question}

\begin{corr} $\RHD$ On $H=\ker\psi$ avec $\psi:\H^1(]0,1[) \to \R$,
$\psi(u):=u(0)-u(1)$, lin\'eaire (\'evident) et continue puisque
\[ |\psi(u)|=|u(0)-u(1)| \leq 2\|u\|_\iy \leq 2\|u\|_{\H^1}. \]
\end{corr}

\begin{Question} On d\'eduit de la question pr\'ec\'edente que $H$,
muni du produit scalaire de $\H^1(]0,1[)$, est \'egalement un espace de Hilbert.
Montrer que toute solution ("forte") $u$ de ce probl\`eme
est \'egalement solution ("faible") de l'\'equation suivante~: quel que soit $v \in H$,
\[ (u|v)_{H^1} = \int_0^1 fv\,\d t. \]
\end{Question}

\begin{corr} $\RHD$ Si $u$ est solution forte, en multipliant par $v \in H$ et en
int\'egrant par parties,
\[ [-u'v]_0^1+(u|v)_{\H^1} = \int_0^1 fv\,\d t, \]
et le terme entre crochets est nul parce-que $u'(0)=u'(1)$ et $v(0)=v(1)$.
\end{corr}

\begin{Question} Montrer qu'on a existence et unicit\'e de solution faible dans $H$.
\end{Question}

\begin{corr} $\RHD$ Il suffit d'appliquer le th\'eor\`eme de Riesz \`a la forme
lin\'eaire continue $\vphi : H \to \R$ d\'efinie par
\[ \vphi(v) := \int_0^1 fv\,\d t. \]
La lin\'earit\'e est \'evidente, et la continuit\'e vient de Cauchy-Schwarz,
\[ |\vphi(v)| \leq \|f\|_{\L^2}\|v\|_{\L^2} \leq \|f\|_{\L^2}\|v\|_{\H^1}. \]
\end{corr}

\begin{Question} Montrer que la solution faible v\'erifie, au sens des distributions,
l'\'e\-qua\-tion suivante dans $\DD'(]0,1[)$~:
\[ -(T_u)''+T_u = T_f. \]
\end{Question}

\begin{corr} $\RHD$ Pour tout $v \in \DD(]0,1[) \subset H$, on a
\[ \int_0^1 u'v'\,\d t + \int_0^1 uv\,\d t = \int_0^1 fv\,\d t, \]
c'est-\`a-dire (noter que $T_{u'}=(T_u)'$ puisque $u$ appartient \`a $\H^1(]0,1[)$)
\[ \la (T_u)',v' \ra + \la T_u,v \ra = \la T_f,v \ra, \]
d'o\`u l'on tire que $-(T_u)''+T_u=T_f$ dans $\DD'(]0,1[)$.
\end{corr}

\begin{Question} On d\'eduit de la question pr\'ec\'edente que la solution est de
classe $\CC^2$ sur $[0,1]$ et qu'elle v\'erifie, pour tout $t \in ]0,1[$,
l'\'equation
\[ -u''(t)+u(t) = f(t). \]
Montrer finalement que cette solution faible est aussi solution forte. 
\end{Question}

\begin{corr} $\RHD$ En int\'egrant par parties (licite car la solution faible est de
classe $\CC^2$), pour tout $v \in H$ on a
\[ 0 = [u'v]_0^1+\int_0^1(\underbrace{-u''+u-f}_{=0})v\,\d t = u'(1)v(1)-u'(0)v(0), \]
d'o\`u l'on tire $u'(0)=u'(1)$ (prendre $v=1$).
Comme $u$ appartient \`a $H$, elle v\'erifie aussi $u(0)=u(1)$.
\end{corr}

\end{Exercice} \vspace*{1em}

\end{document}

\begin{Exercice}

\begin{Question}
\end{Question}

\begin{corr} $\RHD$
\end{corr}

\end{Exercice} \vspace*{1em}
