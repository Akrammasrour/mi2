% Written on Mon  9 Apr 2018 12:02:58 CEST
% by Jean-Baptiste Caillau, LJAD, Univ. Cote d'Azur & CNRS/Inria
\documentclass[11pt,a4paper]{article}
\usepackage{hyperref}
\usepackage{amsmath}
\usepackage{mathrsfs}
\usepackage{graphicx}
\usepackage[latin1]{inputenc}
\usepackage{tp}
\def\N{\mathbf{N}}
\def\Z{\mathbf{Z}}
\def\Q{\mathbf{Q}}
\def\R{\mathbf{R}}
\def\C{\mathbf{C}}
\def\K{\mathbf{K}}
\def\L{\mathrm{L}}
\def\H{\mathrm{H}}
\def\M{\mathrm{M}}
\def\W{\mathrm{W}}
\def\O{\mathrm{O}}
\def\tr{\mathrm{tr}}
\def\ch{\mathrm{ch}}
\def\sh{\mathrm{sh}}
\def\Vect{\mathrm{Vect}}
\def\iy{\infty}
\def\d{\mathrm{d}}
\def\t{\ \!^t\!}
\def\veps{\varepsilon}
\def\vphi{\varphi}
\def\la{\langle}
\def\ra{\rangle}
\def\noi{\noindent}
\def\cf{\emph{cf.}}
\def\ie{\emph{i.e.}}
\renewcommand{\tilde}{\widetilde}
\renewcommand{\hat}{\widehat}

\title{TD~5 -- S\'eries de Fourier}
\shorttitle{TD~5}
\numero{TD~5}
\date{2019--2020}
\discipline{MI2}
\promotion{Polytech Nice Sophia --- MAM3}

\begin{document}
\maketitle

% Exercice 1
\begin{Exercice}[Fonction $\zeta$ de Riemann]
Soit
$$ \zeta(s) = \sum_{n \geq 1} \frac{1}{n^s},\ s \in ]1,+\iy[. $$
Calculer $\zeta(2)$, $\zeta(4)$ et $\zeta(6)$. 
\end{Exercice} \vspace*{1em}

% Exercice 2
\begin{Exercice}
Soit $f \in \R^\R$, paire et $2\pi$-p\'eriodique d\'efinie par
$f(x)=\pi-2x$ sur $[0,\pi[$.
\begin{Question} Donner l'expression de la s\'erie de Fourier de $f$
sur la base hilbertienne des polyn\^omes trigonom\'etriques.
\end{Question}

\begin{Question} Indiquer la nature de la convergence de la s\'erie de
Fourier de $f$. \end{Question}

\begin{Question} En d\'eduire
$$ \sum_{p \geq 0} \frac{1}{(2p+1)^2}\ \text{et}\ 
   \sum_{p \geq 0} \frac{1}{(2p+1)^4}\cdot$$
\end{Question}
\end{Exercice} \vspace*{1em}

% Exercice 3
\begin{Exercice}
Soit $f \in \R^\R$ $2\pi$-p\'eriodique d\'efinie par
$f(x)=e^{ax}$ sur $[0,2\pi[$ ($a \neq 0$).
\begin{Question} Donner l'expression de la s\'erie de Fourier de $f$
sur la base hilbertienne des polyn\^omes trigonom\'etriques.
\end{Question}

\begin{Question} Indiquer la nature de la convergence de la s\'erie de
Fourier de $f$. \end{Question}

\begin{Question} En d\'eduire
$$ \sum_{n \geq 1} \frac{a}{a^2+n^2}\cos nx\ \text{et}\ 
   \sum_{n \geq 1} \frac{n}{a^2+n^2}\sin nx. $$
\end{Question}

\begin{Question} En d\'eduire \'egalement
$$ \sum_{n \geq 1} \frac{1}{(a^2+n^2)^2}\ \text{et}\ 
   \sum_{n \geq 1} \frac{n^2}{(a^2+n^2)^2}. $$
(Rappel~: $\ch\,x=(e^x+e^{-x})/2$ et $\sh\,x=(e^x-e^{-x})/2$.)
\end{Question}
\end{Exercice} \vspace*{1em}

\newpage
% Exercice 4
\begin{Exercice} Soit $E$ l'ensemble des (classes de) fonctions
mesurables $x:[0,1] \to \R$ telles que
$$ \int_{[0,1]} \frac{|x(t)|^2}{t}dt < \iy. $$

\begin{Question} Montrer que $E$ est un espace vectoriel
contenant les fonctions nulles et d\'erivables \`a l'origine.
\end{Question}

\begin{Question} Montrer que
$$ (x|y) = \int_{[0,1]} \frac{x(t)y(t)}{t}dt $$
d\'efinit un produit scalaire sur $E$. \end{Question}

\begin{Question} On note $(P_n)_{n \geq 1}$ le SON obtenu par
orthonormalisation de $\R[X] \backslash \R$. Calculer $P_i$,
$i=1,\dots,3$. \end{Question}
\end{Exercice} \vspace*{1em}

\vfill \begin{flushright}{\footnotesize \emph{En ligne sous}
\texttt{caillau.perso.math.cnrs.fr/mi2}} \end{flushright}

\end{document}
