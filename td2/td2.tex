% Written on Mon Oct 21 00:09:52 CEST 2002
% by Jean-Baptiste Caillau, ENSEEIHT-IRIT, UMR CNRS 5505
\documentclass[11pt,a4paper]{article}
\usepackage{hyperref}
\usepackage{amsmath}
\usepackage{mathrsfs}
%\usepackage{french}
\usepackage{graphicx}
\usepackage{tp}
\def\N{\mathbf{N}}
\def\Z{\mathbf{Z}}
\def\Q{\mathbf{Q}}
\def\R{\mathbf{R}}
\def\C{\mathbf{C}}
\def\Cc{\mathscr{C}}
\def\T{\mathbf{T}}
\def\K{\mathbf{K}}
\def\L{\mathrm{L}}
\def\H{\mathrm{H}}
\def\W{\mathrm{W}}
\def\iy{\infty}
\def\d{\mathrm{d}}
\def\t{\ \!^t\!}
\def\veps{\varepsilon}
\def\vphi{\varphi}
\def\la{\langle}
\def\ra{\rangle}
\def\noi{\noindent}
\def\cf{\emph{cf.}}
\def\ie{\emph{i.e.}}
\renewcommand{\tilde}{\widetilde}
\renewcommand{\hat}{\widehat}
\theoremstyle{plain}
\newtheorem{thrm}{Th\'eor\`eme}[section]
\newtheorem{prpstn}{Proposition}[section]
\newtheorem{lmm}{Lemme}[section]
\newtheorem{crllr}{Corollaire}[section]
\newtheorem{dfntn}{D\'efinition}[section]
\theoremstyle{definition}
\newtheorem{rmrk}{Remarque}[section]

\title{TD~2 -- Th\'eor\`eme de Cauchy-Lipschitz}
\shorttitle{TD~2}
\numero{TD~2}
\date{2019--2020}
\discipline{MI2}
\promotion{Polytech Nice Sophia --- MAM3}

\begin{document}
\maketitle

\stepcounter{section}

\begin{dfntn} \label{def1}
Soit $f:\Omega \subset \R \times \R^n \to \R^n$, $\Omega$
ouvert, $f$ continue, et soit $(t_0,x_0) \in \Omega$. Une solution du
probl\`eme de Cauchy de second membre $f$ et de condition initiale
$(t_0,x_0)$ est un couple $(I,x)$ o\`u $I$ est un intervalle ouvert de $\R$ et $x:I
\to \R^n$ une fonction d\'erivable telle que
$$ \left\{ \begin{array}{l}
  \dot{x}(t)=f(t,x(t)),\ t \in I\\
  x(t_0)=x_0.
\end{array} \right. $$
\end{dfntn}

La d\'efinition pr\'ec\'edente implique en particulier qu'une solution
$(I,x)$ est telle que $t_0 \in I$ (l'intervalle ouvert est donc non-vide), 
que $(t,x(t)) \in \Omega$ quel que soit $t \in I$, et que $x \in 
\Cc^1(I,\R^n)$. On aura besoin de la g\'en\'eralisation suivante du
th\'eor\`eme du point fixe~:\\

\begin{Exercice} Soit $F$ une partie ferm\'ee (et non-vide) d'un espace de Banach $E$,
et soit $g:F \to F$ telle que $g^p$ soit contractante pour un certain naturel $p$.
Montrer que $g$ poss\`ede un unique point fixe.
\end{Exercice} \vspace*{1em}

\begin{Exercice} (Th\'eor\`eme de Cauchy-Lipschitz) Comme \`a la
d\'efinition \ref{def1}, on suppose $f$ continue sur l'ouvert $\Omega$.

\begin{Question} \label{q1}
Montrer que l'ensemble des solutions ordonn\'e par
$$ (I,x) \leq (J,y) \iff I \subset J \text{ et } x=y_{|I} $$
et dont on admet qu'il est non-vide\footnote{Th\'eor\`eme de Peano~: la
seule continuit\'e de $f$ garantit l'existence de solution.}
poss\`ede un \'el\'ement maximal (on parle de \emph{solution maximale} du
probl\`eme de Cauchy).
\end{Question}

On suppose d\'esormais que $f$ est
\emph{localement Lipschitzienne en $x$}, \ie{}
que tout point de $\Omega$ poss\`ede un voisinage $V$ sur lequel $f$ est
Lipschitzienne en $x$~: il existe $k \geq 0$ tel que pour tous $t,x,y$ tels
que $(t,x)$ et $(t,y)$ soient dans $V$,
\begin{equation} \label{eq1}
  |f(t,x)-f(t,y)| \leq k|x-y|.
\end{equation}
Dans (\ref{eq1}), $|.|$ d\'esigne l'une quelconque des normes
\'equivalentes sur $\R^n$.

\begin{Question} Montrer qu'il existe un voisinage $C=B_f(t_0,\eta) \times
B_f(x_0,\veps)$ (ou \emph{cylindre de s\'ecurit\'e}) sur lequel $f$ est
Lipschitzienne en $x$ et tel que $\eta \sup_C |f| \leq \veps$.
\end{Question}

\begin{Question} Soit $E=\Cc^0(B_f(t_0,\eta),\R^n)$ muni de la norme
$\|x\|_\iy = \sup_{B_f(t_0,\eta)} |x|$,
soit $F \subset E$ l'ensemble des fonctions de $E$ \`a valeurs dans $B_f(x_0,\veps)$,
et soit $\phi : F \to F$ d\'efinie par
$$ \phi(x)(t) = x_0 + \int_{t_0}^t f(s,x(s))\,\mathrm{d}s. $$
Montrer que $\phi$ poss\`ede un unique point fixe. En d\'eduire
l'existence d'une solution pour le probl\`eme de Cauchy.
\end{Question}

\begin{Question} Montrer que le probl\`eme poss\`ede une et une seule
solution maximale (au sens de la question \ref{q1}). Montrer que les
courbes int\'egrales maximales forment une partition de $\Omega$.
\end{Question}

\begin{Question} Discuter les trois exemples suivants~: $\dot{x}=x$,
$\dot{x}=x^2$, et $\dot{x}=\sqrt{|x|}$.
\end{Question}
\end{Exercice} \vspace*{1em}

\begin{Exercice} Soit $f : I \times \R \to \R$, $I$ ouvert, $f$ continue~;
on suppose que $f$ est Lipschitzienne en $x$ sur tout $I \times \R$.

\begin{Question}
Montrer que le probl\`eme poss\`ede une
\emph{solution globale} (\ie{} d\'efinie
sur $I$ tout entier).
\end{Question}

\begin{Question} Appliquer la question pr\'ec\'edente au probl\`eme de
Cauchy lin\'eaire
$$ \dot{x}=A(t)x+b(t)$$
avec $A$ et $b$ continues
de $I$ dans $\mathscr{L}(\R^n,\R^n)$ et $\R^n$, respectivement. 
\end{Question}
\end{Exercice} 

\vfill \begin{flushright}{\footnotesize \emph{En ligne sous}
\texttt{caillau.perso.math.cnrs.fr/mi2}} \end{flushright}

\end{document}
