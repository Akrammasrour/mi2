% Written on Tue  6 Feb 2018 22:20:27 CET
% by Jean-Baptiste Caillau, Univ. Cote d'Azur & CNRS/Inria
\documentclass[11pt,a4paper]{article}
\usepackage{hyperref}
\usepackage{amsmath}
\usepackage{mathrsfs}
\usepackage[french]{babel}
\usepackage{graphicx}
\usepackage{tp}
\def\N{\mathbf{N}}
\def\Z{\mathbf{Z}}
\def\Q{\mathbf{Q}}
\def\R{\mathbf{R}}
\def\C{\mathbf{C}}
\def\CC{\mathscr{C}}
\def\T{\mathbf{T}}
\def\K{\mathbf{K}}
\def\L{\mathrm{L}}
\def\H{\mathrm{H}}
\def\W{\mathrm{W}}
\def\M{\mathrm{M}}
\def\O{\mathrm{O}}
\def\Im{\mathrm{Im}}
\def\Vect{\mathrm{Vect}}
\def\Min{\mathrm{Min}}
\def\BV{\mathrm{BV}}
\def\Isom{\mathrm{Isom}}
\def\iy{\infty}
\def\d{\mathrm{d}}
\def\t{\ \!^t\!}
\def\tr{\mathrm{tr}}
\def\veps{\varepsilon}
\def\vphi{\varphi}
\def\la{\langle}
\def\ra{\rangle}
\def\noi{\noindent}
\def\cf{\emph{cf.}}
\def\ie{\emph{i.e.}}
\def\etc{\emph{etc.}}
\renewcommand{\tilde}{\widetilde}
\renewcommand{\hat}{\widehat}
\theoremstyle{plain}
\newtheorem{thrm}{Th\'eor\`eme}[section]
\newtheorem{prpstn}{Proposition}[section]
\newtheorem{lmm}{Lemme}[section]
\newtheorem{crllr}{Corollaire}[section]
\newtheorem{dfntn}{D\'efinition}[section]
\theoremstyle{definition}
\newtheorem{rmrk}{Remarque}[section]

\title{TD~1 -- Calcul diff\'erentiel}
\shorttitle{TD~1}
\numero{TD~1}
\date{2019--2020}
\discipline{MI2}
\promotion{Polytech Nice Sophia --- MAM3}

\begin{document}
\maketitle

% Exo 1
\begin{Exercice}
%% \begin{Question} Montrer que l'application $f : \R \times \R^n \to \R^n$ d\'efinie par
%% $f(\lambda,x):=\lambda x$ est d\'erivable et donner l'expression de sa d\'eriv\'ee.
%% \end{Question}
\begin{Question} Montrer que l'application $f : \R^n \times \R^n \to \R$ d\'efinie par
$f(x,y):=(x|y)$ est d\'erivable et donner l'expression de sa d\'eriv\'ee.
\end{Question}
%% \begin{Question} Soit $f : \R^n \times \R^m \to \R^p$ une application bilin\'eaire.
%% Montrer que $f$ est d\'erivable et donner l'expression de sa d\'eriv\'ee.
%% \end{Question}
\begin{Question} Soient $f : \R^n \to \R$ et $g : \R^n \to \R^m$ deux
applications d\'erivables. 
Montrer que l'application $k : \R^n \to \R^m$ d\'efinie par 
$k(x):=f(x)g(x)$ est d\'erivable et donner l'expression de sa d\'eriv\'ee.
\end{Question}
\end{Exercice}

% Exo 2
\begin{Exercice}
\begin{Question} Montrer que l'application $f : \R^3 \to \R^2$ d\'efinie par
\[ f(x_1,x_2,x_3) := \left[ \begin{array}{r} x_1\cos(x_2 x_3)\\-x_2\sin(x_1 x_3) \end{array} \right] \]
est d\'erivable et donner l'expression de sa d\'eriv\'ee.
\end{Question}
\begin{Question} Montrer que l'application $f : \R^2 \to \R^4$ d\'efinie par
\[ f(x_1,x_2) := \left[ \begin{array}{r} x_1+x_2\\ x_2+x_1^2\\ x_1-x_2^3\\-x_2+x_1^4 \end{array} \right] \]
est d\'erivable et donner l'expression de sa d\'eriv\'ee.
\end{Question}
\end{Exercice}

% Exo 3
\begin{Exercice} Soit $A \in \M(n,\R)$, $b \in \R^n$ et $c \in R$.
\begin{Question}
Montrer que l'application $f : \R^n \to \R$ d\'efinie par
\[ f(x) := \frac{1}{2}(Ax|x)+(b|x)+c \]
est d\'erivable et donner l'expression de son gradient.
\end{Question}
\begin{Question} Montrer que $f$ est deux fois d\'erivable et donner son hessien.
\end{Question}
\end{Exercice}

\newpage

% Exo 4
\begin{Exercice}
\begin{Question}
Soit $A \in \M(m,n,\R)$ et $b \in \R^m$.
Montrer que l'application $f : \R^n \to \R$ d\'efinie par
\[ f(x) := \frac{1}{2} \|Ax-b\|^2 \]
est deux fois d\'erivable et donner l'expression de son gradient et de son hessien.
\end{Question}
\begin{Question}
Soit $F : \R^n \to \R^m$ deux fois d\'erivable.
Montrer que l'application $f : \R^n \to \R$ d\'efinie par
\[ f(x) := \frac{1}{2} \|F(x)\|^2 \]
est deux fois d\'erivable et donner l'expression de son gradient et de son hessien.
\end{Question}
\end{Exercice}

% Exo 5
\begin{Exercice}
Montrer que l'application $f : \R^n\backslash \{0\} \to \R$ d\'efinie par
\[ f(x) := \|x\| \]
est deux fois d\'erivable et donner l'expression de son gradient et de son hessien.
\end{Exercice}

\vfill \begin{flushright}{\footnotesize \emph{En ligne sous}
\texttt{caillau.perso.math.cnrs.fr/mi2}} \end{flushright}

\end{document}
